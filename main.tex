%Préambule du document :
\documentclass[a4paper, 9pt]{article}
\usepackage[left=2cm,right=2cm,top=2cm,bottom=2cm]{geometry} 
\usepackage[utf8]{inputenc}
\usepackage[T1]{fontenc}
\usepackage[french]{babel}
\usepackage{graphicx}
\usepackage{color}   
\usepackage{hyperref}
\usepackage{caption} 
\usepackage[]{Raf_Notations_Torseurs}
\usepackage{pgfgantt}
\usepackage{rpcinematik}
\usepackage{pdfpages}
\usepackage{listings}
\usepackage{xcolor} 
\definecolor{listinggray}{gray}{0.9}
\definecolor{lbcolor}{rgb}{0.9,0.9,0.9}
\definecolor{Darkgreen}{rgb}{0,0.4,0}
\lstset{
	backgroundcolor=\color{lbcolor},
	tabsize=4,    
	%   rulecolor=,
	language=[GNU]C++,
	basicstyle=\scriptsize,
	upquote=true,
	aboveskip={1.5\baselineskip},
	columns=fixed,
	showstringspaces=false,
	extendedchars=false,
	breaklines=true,
	prebreak = \raisebox{0ex}[0ex][0ex]{\ensuremath{\hookleftarrow}},
	frame=single,
	numbers=left,
	showtabs=false,
	showspaces=false,
	showstringspaces=false,
	identifierstyle=\ttfamily,
	keywordstyle=\color[rgb]{0,0,1},
	commentstyle=\color[rgb]{0.026,0.112,0.095},
	stringstyle=\color[rgb]{0.627,0.126,0.941},
	numberstyle=\color[rgb]{0.205, 0.142, 0.73},
	%        \lstdefinestyle{C++}{language=C++,style=numbers}’.
}
\lstset{
	backgroundcolor=\color{lbcolor},
	tabsize=4,
	language=C++,
	captionpos=b,
	tabsize=3,
	frame=lines,
	numbers=left,
	numberstyle=\tiny,
	numbersep=5pt,
	breaklines=true,
	showstringspaces=false,
	basicstyle=\footnotesize,
	%  identifierstyle=\color{magenta},
	keywordstyle=\color[rgb]{0,0,1},
	commentstyle=\color{Darkgreen},
	stringstyle=\color{red}
}

\begin{document}


% configuration des couleurs pour le GANTT
\definecolor{bartermine}{RGB}{38,252,186}
\definecolor{barnontermine}{RGB}{245,69,28}
\definecolor{barvert}{RGB}{154,230,55}
\definecolor{barnonterminegroupe}{RGB}{252,202,73}
\definecolor{colorsoustache}{RGB}{252,202,73}
\definecolor{groupblue}{RGB}{88,123,245}
\definecolor{linkred}{RGB}{245,69,28}
\definecolor{colorsoustache}{RGB}{252,202,73}
%\renewcommand\sfdefault{phv}
%\renewcommand\mddefault{mc}
%\renewcommand\bfdefault{bc}
\setganttlinklabel{s-s}{START-TO-START}
\setganttlinklabel{f-s}{FINISH-TO-START}
\setganttlinklabel{f-f}{FINISH-TO-FINISH}
%\sffamily


\begin{center} % Page de présentation

	
	\textsc{\huge\textbf{{Mini projet de 12h\\}}}
	\vspace{.5cm}
	\textsc{\Large	{Du\\}}
	\vspace{.5cm}
	\textsc{\Large	{Lycée François RABELAIS de Chinon\\}}
	\vspace{.5cm}
	\textsc{\large	{Sciences de l'ingénieur}}
	
	\vspace{.5cm}
	
	\large{\textit	{Présenté par}}\\
	
    \bigskip
	
	\huge{\textbf	{Meriadek \textsc{ACKERMANN}\\					
					\bigskip
					Louane \textsc{PROVOST}\\
					\bigskip
					Loïs \textsc{DURENDEAU}\\
					\bigskip
					Colin \textsc{FARAULT}\\
					\bigskip
					Antony \textsc{MIRIBEL}\\
					\bigskip
					Ronan \textsc{BOUTELOUP}}}\\				
				
					\vspace{.5cm}
					
	\textit{\large{	Thème: Aéronautique\\						
					Support: Gouverne de profondeur}}							
				
	\vspace{.15cm}
	\rule{250pt}{0.3mm}\\
	
	\bigskip

	\Large{\textbf{Problématique : Comment créer une maquette de gouverne de profondeur fonctionnelle avec un retour utilisateur sur écran?}}\\	
	\vspace{.3cm}
	\includegraphics[scale=.3]{illustration}\\	
	
	\vspace{.15cm}
	\rule{250pt}{0.3mm}\\
	\vspace{.15cm}
	
	\textit{\large{Réalisé le 12 mars 2024}}				
	
	\vspace{0.15cm}
	
	\textit{\large{Encadré par :}}\\
	\vspace{.25cm}
	\Large{	\textsc{Mr Guibert} : Professeur de génie électrique}
    	

\end{center}

%Corps du document :
\newpage
	\tableofcontents
\newpage

\section{Organisation du projet}
\subsection{Cahier des charges}
\tiny{~}\\
\normalsize-La carte doit être alimentée en USB et l'alimentation de la carte arduino\\
-La gouverne doit être déplacée par le servo moteur\\
-Le système doit être protégé des court circuits\\
-Les matériaux doivent provenir du laboratoire de Science de L'Ingénieur\\
-La programmation devra être effectuée en C sur une carte arduino\\
-Les matériaux devront respecter du mieux que possible l'environnement\\
-Des informations seront affichée en temps réel via un écran LCD\\
-L'angle d'incidence de l'empanage horizontale doit être obtenable via un potentiomètre\\
\subsection{Gantt}
\begin{ganttchart}[
%Configuration du gantt
    canvas/.append style={fill=none, draw=black!5, line width=.75pt},
    hgrid style/.style={draw=black!5, line width=.75pt},
    vgrid={*1{draw=black!5, line width=.75pt}},
% réglage bar today
%    today=0,
%    today rule/.style={
%      draw=black!64,
%      dash pattern=on 3.5pt off 4.5pt,
%      line width=1.5pt
%    },
    today label font=\small\bfseries,
    title/.style={draw=none, fill=none},
    title label font=\bfseries\footnotesize,
    title label node/.append style={below=7pt},
    include title in canvas=false,
    bar label font=\mdseries\small\color{colorsoustache},
    bar label node/.append style={left=2cm},
    bar/.append style={draw=none, fill=bartermine},
    bar incomplete/.append style={fill=barnonterminegroupe},
    bar progress label font=\mdseries\footnotesize\color{black!70},
    group incomplete/.append style={fill=barnontermine},
    group/.append style={draw=none, fill=bartermine},
    group left shift=0,
    group right shift=0,
    group height=.5,
    group peaks tip position=0,
    group label node/.append style={left=.6cm},
    group progress label font=\bfseries\small,
    link/.style={-latex, line width=1.5pt, linkred},
    link label font=\scriptsize\bfseries,
    link label node/.append style={below left=-2pt and 0pt}
  ]{1}{23}
  \gantttitle[
    title label node/.append style={below left=7pt and -3pt}
  ]{JOURS:\quad9}{1}
  \gantttitlelist{10,...,31}{1} \\
  
%Partie 1 du gantt  
  \ganttgroup[progress=100]{Partie 1: Système électrique}{1}{1} \\
  \ganttbar[
    progress=100,
    name=WBS1B
  ]{\textbf{\footnotesize{Création du schéma}}}{1}{1} \\[grid]
 
%Partie 2 du gantt
  \ganttgroup[progress=100]{Partie 2 : Mécanique}{1}{23} \\
  \ganttbar[progress=100]{\textbf{\footnotesize{Prise des mesures}}}{1}{8} \\
  \ganttbar[progress=100]{\textbf{\footnotesize{Calcul des dimensions}}}{1}{1} \\
  \ganttbar[progress=100]{\textbf{\footnotesize{Modélisation 3D}}}{1}{23} \\
 
  
%Partie 3 du gantt
	\ganttgroup[progress=90]{Partie 3 : Code}{15}{23}\\
	\ganttbar[progress=90]{\textbf{\footnotesize{Création du C++}}}{15}{23} \\
	
%Partie 4 du gantt
  \ganttgroup[progress=80]{Partie 4 : Rendu}{10}{23} \\
  \ganttbar[progress=100]{\textbf{\footnotesize{Présentation orale}}}{4}{5} \\
  \ganttbar[progress=60]{\textbf{\footnotesize{Dossier}}}{6}{8} \\



\end{ganttchart}
\newpage


\section{Schémas \& mise en plan 3d}
	\subsection{Schéma cinématique}
		\begin{center}
			\begin{tikzpicture}[very thick]
				\draw[color=red!60] (0.5,0) -- (0.5,-1);
				\draw[color=red!60] (3.5,0) -- (3.5,-1);
				\draw[color=red!60] (0,-0.5) -- (5,-0.5);
				\draw[color=blue!60](1,0) -- (3,0);
				\draw[color=blue!60] (1,-1) -- (3,-1);
				\draw[color=blue!60] (2,1) -- (2,0);
				\draw[color=red!60] (4.5,0.5) -- (5.5,0.5) (5,0.5) -- (5,-1.5) (4.5,-1.5) -- (5.5,-1.5);
				\draw[color=barvert!60] (4.5,-1.53) -- (5.5,-1.53) (5,-1.53) -- (5,-3.5) (4.5,-3.5) -- (5.5,-3.5) (5, -2.5) -- (10,-2.5) (6.5,-2) -- (6.5,-3) (9.5,-2) -- (9.5,-3);
				\draw[color=black!60] (7,-2) -- (9,-2) (7,-3) -- (9,-3) (8,-3) -- (8,-5);
				\draw[color=blue!60] (6,-5.5) -- (10,-5.5) (6.5,-5) -- (6.5, -6) (9.5, -5) -- (9.5,-6);
				\draw[color=black!60] (7,-5) rectangle (9,-6);
				\draw[color=black, -stealth] (5,-2.5) -- (5, -0.5);
				\node at (5.2,-0.5) (nodez) {z};
				\draw[color=black, -stealth] (5,-2.5) -- (3, -2.5);
				\node at (3,-2.3) (nodey) {y};
			\end{tikzpicture}
		\end{center}
	\subsection{Schéma électrique}
		\includegraphics[scale=.4]{schema-electrique}\\
	\newpage
	\includepdf[scale=.8, pagecommand=\subsection{Mise en plan 3d}]{Empanage-horizontal}
		
\newpage

\section{Programmation}
\subsection{Code C++}
\begin{lstlisting}[language=c++]
	#include <LiquidCrystal.h>   //Importation des bibliotheques necessaires au code
	#include <Servo.h>
	const int POTENTIOMETRE_PIN = A0;           //Declaration du pin du potentiometre/joystick
	const int INCLINAISON_PIN = A1;         //Declaration du pin du potentiometre/gyroscope 
	const float DIVIDER_POT_DEGRES = 3.777778;  //Initialisation de la valeur pour la convertion de la valeur potentiometre en degree
	const int SERVO_PIN = 8;                    //Declaration du pin du servomoteur
	float ordreProfondeur  = 0;                 //Variable pour gerer l'angle du servomoteur
	int anglePotentiometre;                  //Variable de l'angle du potentiometre
	Servo servo;
	// Initialisation des pins de l'ecran lcd (different du cours)
	const int rs = 12, en = 11, d4 = 5, d5 = 4, d6 = 3, d7 = 2;
	LiquidCrystal lcd(rs, en, d4, d5, d6, d7);
	//Setup
	void setup() {	
		InitialiseLCD();	
		//Initialise le moniteur serie à la vitesse de 9600 bauds
		Serial.begin(9600);
		//Le servo est sur le pin 8
		servo.attach(8);
	}
	//Main
	void loop() {
		//Affichage des courbes pour debugger
		AffichageTraceurSerie();
		//Affiche les informations sur l'ecran
		AfficheInfosLCD();
		//Le servo bouge en fonction de la position du joystick
		servo.write(PositionServoDepuisOrdreDegre(ordreProfondeur));
		OrdreProfondeur();
		//Attente de 50 msec
		delay(50);
	}
	int PositionServoDepuisOrdreDegre(float degre) {
		int result round(90+(degre*180/270));
		if(result<0) result = 0;
		if(result>180) result = 180;
		return result;	
	}
	//Initialisation de l'ecran LCD
	void InitialiseLCD() {	
		//Ecran de 20 colonnes x 4 lignes
		lcd.begin(20, 4);
		lcd.setCursor(0, 0);
		lcd.print("joystick :");
		lcd.setCursor(0, 1);
		lcd.print("incidence :");
		lcd.setCursor(0, 3);
		lcd.print("ordre servo :");
	}
	// Affichage des ordres et mesures et du resultat sur le servo
	void AfficheInfosLCD() {
		float incidenceDegre= LitPinEnDegre(POTENTIOMETRE_PIN);
		float inclinaisonDegre = LitPinEnDegre(INCLINAISON_PIN);
		//affichage de l'incidence demandee sur la commande de profondeur
		lcd.setCursor(13, 0);
		lcd.print("       ");
		lcd.setCursor(13, 0);
		lcd.print(incidenceDegre);
		//affichage de l'inclinaison mesuree par le joystick (gyroscope)
		lcd.setCursor(13, 1);
		lcd.print("       ");
		lcd.setCursor(13, 1);
		lcd.print(inclinaisonDegre);
		//affichage de l'ordre envoye au servo
		lcd.setCursor(14, 3);
		lcd.print("      ");
		lcd.setCursor(14, 3);
		lcd.print(PositionServoDepuisOrdreDegre(ordreProfondeur));
	}
	//Determine l'ordre actuel sur la profondeur : soit +X, soit -X
	float OrdreProfondeur() {
		//Lit le joystick
		float ordre = LitPinEnDegre(POTENTIOMETRE_PIN)/25; //Variable l'angle du joystick (divise par 25 pour rendre les deplacements du servo plus fluide)
		//Lit le potentiometre/gyroscope
		float anglePotentiometre = LitPinEnDegre(INCLINAISON_PIN);
		ordreProfondeur+=ordre;
		if(ordreProfondeur<-135) ordreProfondeur = -135;
		if(ordreProfondeur>135) ordreProfondeur = 135;
		else if(anglePotentiometre != ordreProfondeur)
		ordreProfondeur = -anglePotentiometre;
	}
	//Lit une PIN et renvoie une valeur en degre
	float LitPinEnDegre(int pin) {
		float inclinaisonDegre = analogRead(pin)/DIVIDER_POT_DEGRES;
		float result= inclinaisonDegre-135;
		if(result > -2 && result <2)
		result = 0;
		return result;
	}
	//Au travers du traceur serie, affiche les courbes du joystick et de l'inclinaison de l'avion
	void AffichageTraceurSerie() {
		Serial.print(LitPinEnDegre(POTENTIOMETRE_PIN));
		Serial.print(",");
		Serial.println(LitPinEnDegre(INCLINAISON_PIN));
	}
\end{lstlisting}
\section{Conclusion}
	%Veillez à insérer un \hspace{0.5cm} avant votre conclusion si elle n'est pas la première
	Ce projet nous à appris à travailler en équipe et à nous répartir les tâches, de plus il nous a amené à utiliser des logiciels que nous ne connaissions pas, notamment Gantt Project, le seul logiciel que je connaisse qui requiert le jre-fx et ne fonctionne pas avec le jdk.
	\begin{flushright}
		-Ronan
	\end{flushright}
	\hspace{0.5cm} Ce projet de commande de vol électrique a été bénéfique du moins personnellement, elle nous a appris à travailler en équipe (de manière plus avancée), à faire des recherches par soi même pour developper de nouvelles capacités (notamment sur le codage) et à s'organiser sur la longueur.\\
	Ce projet malgré sa difficulté a été éducative et les compétences développées nous resserviront sûrement.
	\begin{flushright}
	 	-Anthony
	\end{flushright}
\end{document} 